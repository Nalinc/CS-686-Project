\def\year{2017}\relax
%File: formatting-instruction.tex
\documentclass[letterpaper] {article} %DO NOT CHANGE THIS
\usepackage{aaai17}  %Required
\usepackage{times}  %Required
\usepackage{helvet}  %Required
\usepackage{courier}  %Required
\usepackage{url}  %Required
\usepackage{graphicx}  %Required
\usepackage{csquotes} %Added from CSCW Paper
\usepackage{siunitx}
\usepackage{comment}
\frenchspacing  %Required
\setlength{\pdfpagewidth}{8.5in}  %Required
\setlength{\pdfpageheight}{11in}  %Required
%PDF Info Is Required:
  \pdfinfo{
/Title			
/Author (AAAI Press Staff)}
\setcounter{secnumdepth}{0}  
 \begin{document}
% The file aaai.sty is the style file for AAAI Press 
% proceedings, working notes, and technical reports.
%
\title{ \\Intelligence behind building conversational bots}
\author{Nalin Chhibber\\ Student ID: 20715659, MMath CS(HCI)\\
}
\maketitle
\begin{abstract}
The main aim of this paper is to describe various techniques that have been used to build such voice interfaces and cover a sizable body of literature related to the area. Additionally, this paper implements RBCB: a conversational bot for casual chat. RBCB is trained on a limited dataset of manually constructed sentences and uses Long Short Term Memory to remember the context. 

\end{abstract}

\section{Introduction}
%\begin{figure}[ht]
%\centering
 % \includegraphics[width=0.8\columnwidth]{figures/sample.jpg}
%\caption{Lorem ipsum dolor sit amet, consectetur adipiscing elit, sed do eiusmod tempor incididunt ut labore et dolore magna aliqua. Ut enim ad minim \textit{veniam}, quis nostrud exercitation ullamco laboris }
%\label{fig:workflow}
%\end{figure}
There has been a lot of optimism in the thought that near future will witness a rapid growth in human-computer-interaction using voice. Unlike other modes of interaction with computers where humans need to adapt with the interface(touch/type), voice interface comes natural to humans and hence can be used as one of the most promising medium to engage them in a productive interaction. This has not only led to an increased demand of conversational agents but also shown an increase in various chatbot development frameworks. Brands are increasingly using chatbots to engage their customers. Within just a couple years, we have seen a different evolution in the design of conversational agents from chatbots in Facebook Messenger, to Siri in iphones, to Microsoft's Cortana, Google Home and Amazon Alexa. 



retrieval, generative
online, batch
end to end vs distributed
word embeddedings
tokenization, stemming and lemmatization
tf-idf
recurrent neural networks
vanishing gradient
lstm
seq2seq and dual encoder 

\section{Classification of conversational agents}
On a really broad scale, task of building a conversational bot can be achieved using a retrival-based model or generative model, or a combination of both. Retrieval-based models are those which have a repository of pre-defined responses to answer a user utterance. Alicebot and Cleverbot are examples of such types. On the other hand, generative models can generate responses they have never seen before. Microsoft's Tay bot is an example of generative model. 


\section{Related work}


\section{Neural Networks}

\subsection{Recurrent Neural Networks} 

\subsection{Long Short Term Memory}

\section{Future Work}
Personality
Domain specific usage

\section{Discussion}


\bibliographystyle{aaai}
\bibliography{ta}
\end{document}

